\documentclass[11pt]{report}

\usepackage{colortbl}
\usepackage{url}

\addtolength{\oddsidemargin}{-1in}
\addtolength{\evensidemargin}{-1in}
\addtolength{\textwidth}{2in}

\addtolength{\topmargin}{-1.0in}
\addtolength{\textheight}{2.0in}

\title{Complete Noob}
\author{Elly (\url{elly@leptoquark.net})}
\date{\today}

\begin{document}

\newcommand{\btitle}[0]{\textit{Complete Noob}}
\newcommand{\dandd}[0]{{\sc Dungeons \& Dragons}}
\newcommand{\tblbg}[0]{\rowcolor[rgb]{0.9,0.9,1.0}}

\maketitle
\tableofcontents

\chapter{Introduction}
The \btitle book is a netbook expansion for the \dandd Roleplaying Game. It is
primarily a nonplayer resource focusing on new options and expanded rules for
D\&D nonplayers who want to create or advance noobs.

\section{Noobs}
Everyone recognizes the Noob. The people filling the town market and giving it
an appropriately urban atmosphere are noobs. The goblins living in the lost
temple just over that hill are probably noobs; the guy cleaning glasses behind
the bar almost certainly is. The distinguishing characteristic of Noobs is their
role: Noobs function as scenery, rather than combat threats or challenges of
skill. As such, Noobs are almost invariably vastly less powerful than PCs or
even non-Noob NPCs, and tend to die en masse whenever real power is present.

\subsection{Why Noobs?}
Given that Noobs fill no interesting function, why do they exist at all? The
answer lies in their vast numbers. The individual Noob is useless, but together,
they vastly outnumber all the non-Noobs in the world put together. Noobs are
responsible for mundane things like growing crops, producing equipment,
defending towns against monsters when non-Noobs are not present, trading goods,
and so on. Quite simply, without Noobs, there would be no food production, and
no towns, and therefore no inns; without inns, there is no place for a party of
PCs to meet for the first time and decide to adventure, and there is no location
for a non-Noob NPC to meet said PCs and enlist them for a dangerous but
potentially lucrative quest. There are also entire industries - splint mail
production, for example - whose existence relies on the continued presence of
Noobs.

Noobiness is a recessive genetic trait - many a powerful PC or NPC was born to
Noob parents, but Noob children are rarely born to non-Noob parents. A peculiar
form of 'reverse inheritance' is currently under study among prominent
Noobologists: when a non-Noob is born to Noob parents, said parents appear to
begin losing some of their Noob traits. This effect is especially pronounced
when the child is a PC.

\subsection{Non-Noobs}
Non-Noobs are PCs, powerful NPCs, and high-level monsters. Non-Noobs tend to be
adventurers, rulers, and military leaders. Non-Noobs occasionally appear in the
guise of Noobs; this most frequently occurs in places of commerce and in inns.
There is a strong statistical correlation between non-Noobs stealing for fun and
profit (as opposed to out of necessity) and the sudden and unexpected appearance
of equal and opposite non-Noobs in certain worlds. This phenomenon is known as
"karmic rebalancing". In general, while non-Noobs are in the driver's seat,
without the Noobs, there would be no cart and no horses.

For more detail about non-Noobs, consult The \textit{Complete Motherfucker}
Handbook.

\chapter{Expanded Classes}
This chapter presents alternate classes for each of the NPC classes given in the
\textit{Dungeon Master's Guide}. These classes are designed to give a different
flavor to Noobs (figuratively, for the benefit of PCs, and literally, for the
benefit of dragons), as well as make them more suitable for certain roles.

\section{Adept}
\subsection{Attuned}
The Attuned is a spontaneous spellcaster, like the sorceror, but without the
deeply ingrained command over magical energies that the sorceror wields, nor the
sheer charisma to truly bend magic to her will. However, the Attuned does have
a small degree of arcane power, and can thus wield a small selection of spells.
\\
\textbf{Hit Die:} d6. \\
The Attuned's class skills are: Concentration (Con), Craft (Int), Knowledge
(any) (Int), Profession (Wis), Spellcraft (Int). \\
\textbf{Skill Points per Level:} 2.

\begin{tabular}{llcccccccccc}
\hline
% ugh, nasty headers
NPC & Base & Fort & Ref & Will & & \multicolumn{6}{c}{Spells per Day} \\
Level & Attack Bonus & Save & Save & Save & Special & 0 & 1st & 2nd & 3rd & 4th
& 5th \\
\hline
       1st & +0 & +0 & +0 & +2 & & 3 & 1 & - & - & - & - \\
\tblbg 2nd & +1 & +0 & +0 & +3 & & 4 & 1 & - & - & - & - \\
       3rd & +1 & +1 & +1 & +3 & Scribe Scroll & 4 & 2 & - & - & - & - \\
\tblbg 4th & +2 & +1 & +1 & +4 & & 4 & 2 & 1 & - & - & - \\
       5th & +2 & +1 & +1 & +4 & Arcane Learning & 4 & 3 & 1 & - & - & - \\
\tblbg 6th & +3 & +2 & +2 & +5 & & 4 & 3 & 2 & - & - & - \\
       7th & +3 & +2 & +2 & +5 & & 4 & 3 & 3 & - & - & - \\
\tblbg 8th & +4 & +2 & +2 & +6 & & 4 & 3 & 3 & 1 & - & - \\
       9th & +4 & +3 & +3 & +6 & & 4 & 4 & 3 & 1 & - & - \\
\tblbg 10th & +5 & +3 & +3 & +7 & Arcane Learning & 4 & 4 & 3 & 2 & - & - \\
       11th & +5 & +3 & +3 & +7 & & 4 & 4 & 4 & 3 & - & - \\
\tblbg 12th & +6/+1 & +4 & +4 & +8 & & 4 & 4 & 4 & 3 & 1 & - \\
       13th & +6/+1 & +4 & +4 & +8 & & 4 & 4 & 4 & 4 & 2 & - \\
\tblbg 14th & +7/+2 & +4 & +4 & +9 & & 4 & 4 & 4 & 4 & 3 & - \\
       15th & +7/+2 & +5 & +5 & +9 & Arcane Learning & 4 & 4 & 4 & 4 & 4 & - \\
\tblbg 16th & +8/+3 & +5 & +5 & +10 & & 4 & 4 & 4 & 4 & 4 & 1 \\
       17th & +8/+3 & +5 & +5 & +10 & & 4 & 4 & 4 & 4 & 4 & 2 \\
\tblbg 18th & +9/+4 & +6 & +6 & +11 & & 4 & 4 & 4 & 4 & 4 & 3 \\
       19th & +9/+4 & +6 & +6 & +11 & & 4 & 4 & 4 & 4 & 4 & 4 \\
\tblbg 20th & +10/+5 & +6 & +6 & +12 & Arcane Mastery & 4 & 4 & 4 & 4 & 4 & 4 \\
\hline
\end{tabular}

\subsubsection{Class Features}
All of the following are class features of the Attuned class. \\
\textbf{Weapon and Armor Proficiency:} Attuned are skilled with all simple
weapons. Attuned are not proficient with any kinds of armor or shields. \\
\textbf{Spells:} An Attuned casts arcane spells (the same type as available to
the sorceror and wizard) drawn from the list given below. An Attuned need not
prepare her spells in advance, and may cast any spell from her class list
spontaneously. The key statistic of an Attuned for spellcasting is Int. \\
\textbf{Scribe Scroll:} At 3rd level, an Attuned gains the Scribe Scroll feat
for free. \\
\textbf{Arcane Learning:} At 5th level, an Attuned may add a single spell of any
level she can cast from the Sorceror/Wizard list to her list of spells. For
example, a 5th level Attuned could add Invisiblity (level 2) to her list, but could not
add Fly (level 3). At 10th level, she may add an additional spell of up to 3rd
level from the Sorceror/Wizard list. At 15th level, she may add an additional
spell of up to 4th level from the Sorceror/Wizard list. \\
\textbf{Arcane Mastery:} At 20th level, an Attuned may add any two spells from
the Sorceror/Wizard list of up to 5th level to her list.

\subsection{Eldritch Spark}
The Eldritch Spark is an Eldritch spellcaster, like the Warlock, but without the
Warlock's powerful command of the eldritch energies that power her abilities. \\

Eldritch Sparks learn fewer invocations than warlocks and can use them for only
a certain number of times per day. Their Eldritch Blast is also less powerful.
\\
\textbf{Hit Die:} d6. \\
The Eldritch Spark's class skills are: Bluff (Cha), Concentration (Con), Craft
(Int), Disguise (Cha), Intimidate (Cha), Jump (Str), Knowledge (arcana) (Int),
Knowledge (the planes) (Int), Knowledge (religion) (Int), Profession (Wis),
Sense Motive (Wis), Spellcraft (Int), and Use Magic Device (Cha). \\
\textbf{Skill Points per Level:} 2 + Int Modifier.

\begin{tabular}{llcccccc}
\hline
% ugh, nasty headers
NPC & Base & Fort & Ref & Will & & Invocations & Invocations \\
Level & Attack Bonus & Save & Save & Save & Special & Known & Per Day \\
\hline

	1st &    +0 & +0 & +0 & +2 & Eldritch Blast (1d6) & 1/0/0/0 & 2/0/0/0 \\
\tblbg  2nd &    +1 & +0 & +0 & +3 & Detect Magic & 1/0/0/0 & 3/0/0/0 \\
	3rd &    +2 & +1 & +1 & +3 & - & 1/0/0/0 & 4/0/0/0 \\
\tblbg	4th &    +3 & +1 & +1 & +4 & - & 2/0/0/0 & 4/0/0/0 \\
	5th &    +3 & +1 & +1 & +4 & Eldritch Blast (2d6) & 2/0/0/0 & 5/0/0/0 \\
\tblbg	6th &    +4 & +2 & +2 & +5 & - & 2/0/0/0 & 6/0/0/0 \\
	7th &    +5 & +2 & +2 & +5 & - & 3/0/0/0 & 6/0/0/0 \\
\tblbg	8th & +6/+1 & +2 & +2 & +6 & - & 3/0/0/0 & 7/0/0/0 \\
	9th & +6/+1 & +3 & +3 & +6 & Eldritch Blast (3d6) & 3/0/0/0 & 8/0/0/0 \\
\tblbg	10th & +7/+2 & +3 & +3 & +6 & - & 3/1/0/0 & 8/2/0/0 \\
	11th & +8/+3 & +3 & +3 & +7 & - & 3/1/0/0 & 8/3/0/0 \\
\tblbg	12th & +9/+4 & +4 & +4 & +8 & - & 3/1/0/0 & 8/4/0/0 \\
	13th & +9/+4 & +4 & +4 & +8 & Eldritch Blast (4d6) & 3/2/0/0 & 8/4/0/0 \\
\tblbg	14th & +10/+5 & +4 & +4 & +9 & - & 3/2/0/0 & 8/5/0/0 \\
	15th & +11/+6/+1 & +5 & +5 & +9 & - & 3/2/0/0 & 8/6/0/0 \\
\tblbg	16th & +12/+7/+2 & +5 & +5 & +10 & - & 3/3/0/0 & 8/6/0/0 \\
	17th & +12/+7/+2 & +5 & +5 & +10 & Eldritch Blast (5d6) & 3/3/0/0 & 8/7/0/0 \\
\tblbg	18th & +13/+8/+3 & +6 & +6 & +11 & - & 3/3/0/0 & 8/8/0/0 \\
	19th & +14/+9/+4 & +6 & +6 & +11 & - & 3/3/1/0 & 8/8/2/0 \\
\tblbg	20th & +15/+10/+5 & +6 & +6 & +12 & - & 3/3/1/0 & 8/8/3/0 \\
\hline
\end{tabular}

\subsubsection{Class Features}
All of the following are class features of the Eldritch Spark class. \\
\textbf{Armored Casting:} Eldritch Sparks can wear light armor without any
spellcasting penalties. \\
\textbf{Weapon and Armor Proficiency:} Eldritch Sparks are skilled with all
simple weapons and light armor. \\
\textbf{Invocations:} The Eldritch Spark gains invocations from the Warlock
list. Save DCs are charisma-based. The warlock may cast up to the listed number
of invocations of each type per day, which need not be chosen in advance. \\
\textbf{Eldritch Blast:} A ray with range 60 feet that deals the listed amount
of damage. Some invocations also affect the Eldritch Spark's Eldritch Blast; see
\textbf{The Complete Arcane} for a list.

\subsection{Godtouched}
The Godtouched is a spontaneous divine spellcaster, much like the standard Druid
or Cleric, but without the ability to summon divine favor to herself at will;
perhaps the Godtouched's god demands more of her before providing a spell, or
perhaps the energy the Godtouched relies on is weaker. Either way, the
Godtouched can only draw on the bulk of her power under certain conditions - in
a certain temple, in a certain grove, within a mile of a particular waterfall,
for example. \\

\textbf{Hit Die:} d6. \\
The Godtouched's class skills are: Concentration (Con), Craft (Int), Knowledge
(religion) (Int), Profession (Wis), Spellcraft (Int). \\
\textbf{Skill Points per Level:} 2.

\begin{tabular}{llcccccccccc}
\hline
NPC & Base & Fort & Ref & Will & & \multicolumn{6}{c}{Spells per Day} \\
Level & Attack Bonus & Save & Save & Save & Special & 0 & 1st & 2nd & 3rd & 4th
& 5th \\
\hline
	1st & +0 & +0 & +0 & +2 & Focus Location & 3 & 1 & - & - & - & - \\
\tblbg	2nd & +1 & +0 & +0 & +3 & - & 4 & 1 & - & - & - & - \\
	3rd & +1 & +1 & +1 & +3 & Brew Potion & 4 & 2 & - & - & - & - \\
\tblbg	4th & +2 & +1 & +1 & +4 & Spell Freedom 0 & 4 & 2 & 1 & - & - & - \\
	5th & +2 & +1 & +1 & +4 & - & 4 & 3 & 1 & - & - & - \\
\tblbg	6th & +3 & +2 & +2 & +5 & - & 4 & 3 & 2 & - & - & - \\
	7th & +3 & +2 & +2 & +5 & - & 4 & 3 & 3 & - & - & - \\
\tblbg	8th & +4 & +2 & +2 & +6 & Spell Freedom 1 & 4 & 3 & 3 & 1 & - & - \\
	9th & +4 & +3 & +3 & +6 & - & 4 & 4 & 3 & 1 & - & - \\
\tblbg	10th & +5 & +3 & +3 & +7 & - & 4 & 4 & 3 & 2 & - & - \\
	11th & +5 & +3 & +3 & +7 & Second Focus Location & 4 & 4 & 4 & 3 & - & - \\
\tblbg	12th & +6/+1 & +4 & +4 & +8 & Spell Freedom 2 & 4 & 4 & 4 & 3 & 1 & - \\
	13th & +6/+1 & +4 & +4 & +8 & - & 4 & 4 & 4 & 4 & 2 & - \\
\tblbg	14th & +7/+2 & +4 & +4 & +9 & - & 4 & 4 & 4 & 4 & 3 & - \\
	15th & +7/+2 & +5 & +5 & +9 & - & 4 & 4 & 4 & 4 & 4 & - \\
\tblbg	16th & +8/+3 & +5 & +5 & +10 & Spell Freedom 3 & 4 & 4 & 4 & 4 & 4 & 1 \\
	17th & +8/+3 & +5 & +5 & +10 & - & 4 & 4 & 4 & 4 & 4 & 2 \\
\tblbg	18th & +9/+4 & +6 & +6 & +11 & - & 4 & 4 & 4 & 4 & 4 & 3 \\
	19th & +9/+4 & +6 & +6 & +11 & - & 4 & 4 & 4 & 4 & 4 & 4 \\
\tblbg	20th & +10/+5 & +6 & +6 & +12 & Spell Freedom 4 & 4 & 4 & 4 & 4 & 4 & 4 \\
\hline
\end{tabular}

\subsubsection{Class Features}
All of the following are class features of the Godtouched. \\
\textbf{Weapon and Armor Proficiency:} Godtouched are skilled wih all simple
weapons, and with light armor. \\
\textbf{Spells:} A Godtouched casts Divine spells (the same type as available to
the cleric or druid), drawn from either the cleric or druid lists (her choice).
A Godtouched need not prepare her spells in advance, and may cast any spell from
her class list spontaneously. The key statistic of a Godtouched for spellcasting
is Wis. \\
\textbf{Focus Location:} At 1st level, the Godtouched must choose a Focus
Location which is an area in which she can freely speak to her deity, which must
be no more than 1000 square feet in area. By default, the Godtouched may only
cast spells inside her Focus Location. At 11th level, the Godtouched may
designate a second Focus Location, which functions identically to her first. \\
\textbf{Brew Potion:} At 3rd level, a Godtouched gains the Brew Potion feat for
free. \\
\textbf{Spell Freedom:} As she grows in power, a Godtouched may learn to
preserve some of her power outside her Focus Locations. She may cast spells
ouside of her Focus Locations, but such spellcastings consume a spell slot one
level higher than the spell \textbf{in addition} to the slot that would normally
be consumed. 

\section{Aristocrat}
\subsection{Monarch}

\section{Commoner}
\textit{It's difficult to know what to put here, and I'm still thinking about
it. Commoners are almost by definition uninteresting, since if they were
interesting they'd be experts or warriors or adepts.}

\section{Expert}
\subsection{Dedicated Specialist}
The Dedicated Specialist is the master of his art, highly trained (or
experienced) in one thing and one thing alone. The Dedicated Specialist is more
focused than the expert in exchange for larger bonuses. \\
\textbf{Hit Die:} d6. \\
The Dedicated Specialist's class skills are: Profession (Wis), plus any four of
his choice.
\textbf{Skill Points per Level:} 6 + Int Modifier.

\begin{tabular}{llcccc}
\hline
NPC & Base & Fort & Ref & Will & \\
Level & Attack Bonus & Save & Save & Save & Special \\
\hline

	1st & +0 & +0 & +0 & +2 & Expertise 1 \\
\tblbg	2nd & +1 & +0 & +0 & +3 & - \\
	3rd & +1 & +1 & +1 & +3 & - \\
\tblbg	4th & +2 & +1 & +1 & +4 & Expertise 2 \\
	5th & +2 & +1 & +1 & +4 & - \\
\tblbg	6th & +3 & +2 & +2 & +5 & - \\
	7th & +3 & +2 & +2 & +5 & Expertise 3 \\
\tblbg	8th & +4 & +2 & +2 & +6 & - \\
	9th & +4 & +3 & +3 & +6 & - \\
\tblbg	10th & +5 & +3 & +3 & +6 & Expertise 4 \\
	11th & +5 & +3 & +3 & +7 & - \\
\tblbg	12th & +6/+1 & +4 & +4 & +8 & - \\
	13th & +6/+1 & +4 & +4 & +8 & Expertise 5 \\
\tblbg	14th & +7/+2 & +4 & +4 & +9 & - \\
	15th & +7/+2 & +5 & +5 & +9 & - \\
\tblbg	16th & +8/+3 & +5 & +5 & +10 & Expertise 6 \\
	17th & +8/+3 & +5 & +5 & +10 & - \\
\tblbg	18th & +9/+4 & +6 & +6 & +11 & - \\
	19th & +9/+4 & +6 & +6 & +11 & Expertise 7 \\
\tblbg	20th & +10/+5 & +6 & +6 & +12 & - \\
\hline
\end{tabular}

\subsubsection{Class Features}
All of the following are class features of the Dedicated Specialist class. \\
\textbf{Weapon and Armor Proficiency:} Dedicated Specialists are proficient with
simple weapons and light armor only. \\
\textbf{Expertise:} As he levels, a Dedicated Specialist gets progressively
better at those few skills he practices most; when he makes a skill check for a
class skill, he may add his Expertise to his roll as a competence bonus.

\section{Warrior}
\textit{The combat mechanics of d20 provide enough variety that actually having
separate 'archer'/'guard'/etc classes doesn't really seem necessary - you can
just outfit a warrior with the right feats and weaponry.}

\end{document}
